

%\gls{<label>}
%\glspl{<label>}
%\Gls{<label>}
%\Glspl{<label>}

%\acrfull{<label>}
%\acrlong{<label>}
%\acrshort{<label>}


\newglossaryentry{abstract test case}
{
    name=abstract test case,
    description={A finite sequence of actions in a FTS such that there exists a sequence of transitions in this FTS, labelled with the corresponding actions}
}

\newglossaryentry{bug}
{
    name=bug,
    description={The result, during the execution of a system, of an error made by the programmer. If it propagates to the output of the system, it causes a failure}
}

\newglossaryentry{concretization}
{
    name=concretization,
    description={Process used to transform abstract test cases into executable test scripts}
}

\newglossaryentry{coverage criterion}
{
    name=coverage criterion,
    description={Criterion defined over a model or a piece of source code and used to drive a test case selection process}
}

\newglossaryentry{error}
{
    name=error,
    description={Mistake made by a programmer during the writing of the source code of a system}
}

\newglossaryentry{failure}
{
    name=failure,
    description={Result (at runtime) of the propagation of a bug to the output of a system}
}

\newglossaryentry{fault}
{
    name=fault,
    description={Synonym for bug}
    see={bug}
}

\newglossaryentry{feature model}
{
    name=feature model,
    description={A model used to represent all the valid products of a product line \cite{Kang1990}. Usually, feature models are represented using a tree structure where features are decomposed into sub-features}
}

\newglossaryentry{feature expression}
{
    name=feature expression,
    description={A boolean expression over features}
}

\newglossaryentry{featured mutant model}
{
    name=featured mutant model,
    description={A compact formalism to represent mutants of a system as a family (\ie a product line)}
}

\newglossaryentry{featured transition system}
{
    name=featured transition system,
    description={A compact formalism used to represent the behaviour of a software product line \cite{Classen2013b}. An FTS is a LTS where transitions are tagged with feature expressions specifying which products may fire the transition}
}

\newglossaryentry{higer-order mutant}
{
    name=higher-order mutant,
    description={Mutant generated from another mutant}
}

\newglossaryentry{killed mutant}
{
    name=killed mutant,
    description={Mutant that has been detected by at least one test case of a test suite}
}

\newglossaryentry{labelled transition system}
{
    name=labelled transition system,
    description={Formalism used to represent the behaviour of a system as a set of states and transitions labelled with actions}
}

\newglossaryentry{live mutant}
{
    name=live mutant,
    description={Mutant that has bot been detected by any test case of a test suite}
}

\newglossaryentry{mutant}
{
    name=mutant,
    description={Faulty system generated by applying a mutation operator on the original system}
}

\newglossaryentry{negative abstract test case}
{
    name=negative abstract test case,
    description={Abstract test case that cannot be executed on the FTS of the product line}
}

\newglossaryentry{positive abstract test case}
{
    name=positive abstract test case,
    description={Abstract test case that can be executed on the FTS of the product line}
}

\newglossaryentry{soda vending machine}
{
    name=soda vending machine,
    description={A case study representing a beverage vending machine product line that sells soda and/or tea (see Section \ref{sec:casestudy:svm} for the complete description)}
}

\newglossaryentry{test suite}
{
    name=test suite,
    description={A set of test cases, selected in order to satisfy a given criterion}
}

\newglossaryentry{usage model}
{
    name=usage model,
    description={Model representing the usage of a system as a \acrfull{DTMC}}
}

%%%%%%%%%%%%%%%%%%%%%%%%%%%%%%%%
% Acronyms 
%%%%%%%%%%%%%%%%%%%%%%%%%%%%%%%%

\newacronym{AbsCon}{AbsCon}{Abstract test case Concretizer}

\newacronym{AGE}{AGE}{\textit{Assembl\'ee G\'en\'erale des \'Etudiants}}

\newacronym{ALE}{ALE}{Automata Language Equivalence}

\newacronym{API}{API}{Application Programming Interface}

\newacronym{BDD}{BDD}{Binary Decision Diagram}

\newacronym{BS}{BS}{Biased Simulation}

\newacronym{CIT}{CIT}{Combinatorial Interaction Testing}

\newacronym{CMS}{CMS}{Content Management System}

\newacronym{CNF}{CNF}{Conjunctive Normal Form}

\newacronym{CCS}{CCS}{Calculus of Communicating Systems}

\newacronym[see={[see:]{usage model}}]{DTMC}{DTMC}{Discrete-Time Markov Chain}

\newacronym{EMP}{EMP}{Equivalent Mutants Problem}

\newacronym[see={[see:]{feature model}}]{FM}{FM}{Feature Model\glsadd{feature model}}

\newacronym[see={[see:]{featured mutant model}}]{FMM}{FMM}{Featured Mutants Model\glsadd{featured mutant model}}

\newacronym{FODA}{FODA}{Feature Oriented Domain Analysis}

\newacronym[see={[see:]{featured transition system}}]{FTS}{FTS}{Featured Transition System\glsadd{featured transition system}}

\newacronym[see={[see:]{labelled transition system}}]{LTS}{LTS}{Labelled Transition System\glsadd{labelled transition system}}

\newacronym{MBT}{MBT}{Model-Based Testing}

\newacronym{MHML}{MHML}{Modal Hennessy-Milner Logic}

\newacronym{MTS}{MTS}{Modal Transition System}

\newacronym{PIN}{PIN}{Personal Identification Number}

\newacronym{PL-CCS}{PL-CCS}{Product Line CCS}

\newacronym{POM}{POM}{Project Object Model}

\newacronym{QTaste}{QTaste}{QSpin Tailored Automated System Test Environment}

\newacronym{RS}{RS}{Random Simulation}

\newacronym{SAT}{SAT}{Boolean Satisfiability Problem}

\newacronym{SM}{SM}{Strong Mutation}

\newacronym{SPL}{SPL}{Software Product Line}

\newacronym{SUT}{SUT}{System Under Test}

\newacronym{TVL}{TVL}{Text-based Variability Language}

\newacronym[see={[see:]{usage model}}]{UM}{UM}{Usage Model}

\newacronym{UIDL}{UIDL}{User Interface Description Language}

\newacronym{VIBeS}{VIBeS}{Variability Intensive Behavioural teSting}

\newacronym{WM}{WM}{Weak Mutation}

